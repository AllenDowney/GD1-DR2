% This file is part of the stream_information project.
% Copyright 2017 the authors. All rights reserved.

% # style notes
% - it is Cram\'er--Rao not Cram\'er-Rao. And yet Fisher-matrix not Fisher--matrix.

% TODO:
% - Reference for the dustmaps package? See software section.

% Story:
% - progenitor - show std_v, std_phi2
% - 2 gaps, 1 under-density - epicycles/streakline? percentiles in polynomial
% - spur - encounter?

\documentclass[modern]{aastex62}

\usepackage{amsmath}

% typography
\setlength{\parindent}{1.\baselineskip}
\newcommand{\acronym}[1]{{\small{#1}}}
\newcommand{\package}[1]{\textsl{#1}}
\newcommand{\gaia}{\textsl{Gaia}}
\newcommand{\pans}{\textsl{Pan-STARRS}}
\newcommand{\DR}{\acronym{DR2}}
\newcommand{\msun}{\textrm{M}_\odot}
\newcommand{\kpc}{\textrm{kpc}}
\newcommand{\mag}{\textrm{mag}}
\newcommand{\kms}{\ensuremath{\textrm{km}~\textrm{s}^{-1}}}
\newcommand{\bs}[1]{\boldsymbol{#1}}
\newcommand{\masyr}{\ensuremath{\textrm{mas}~\textrm{yr}^{-1}}}
\newcommand{\feh}{\ensuremath{[\textrm{Fe} / \textrm{H}]}}
\newcommand{\article}{\textsl{Letter}}

\newcommand{\todo}[1]{{\color{red} TODO: #1}}

% aastex parameters
% \received{not yet; THIS IS A DRAFT}
%\revised{not yet}
%\accepted{not yet}
% % Adds "Submitted to " the arguement.
% \submitjournal{ApJ}
\shorttitle{GD-1 in Gaia DR2}
\shortauthors{Price-Whelan \& Bonaca}

%@arxiver{}

\begin{document}\sloppy\sloppypar\raggedbottom\frenchspacing % trust me

\title{Off the beaten path: \\
       Gaia reveals GD-1 stars perturbed off of the main stream track}
% Gaia DR2 reveals stream stars...
% A first look at the GD-1 stellar stream with Gaia DR2

\author[0000-0003-0872-7098]{Adrian~M.~Price-Whelan}
\altaffiliation{These authors contributed equally to this work.}
\affiliation{Department of Astrophysical Sciences,
             Princeton University, Princeton, NJ 08544, USA}
\email{adrn@astro.princeton.edu}
\correspondingauthor{Adrian M. Price-Whelan}

\author[0000-0002-7846-9787]{Ana Bonaca}
\altaffiliation{These authors contributed equally to this work.}
\affil{Harvard--Smithsonian Center for Astrophysics, Cambridge, MA 02138, USA}

\begin{abstract}\noindent % trust me
Tidally-disrupted globular clusters are transformed into thin, dynamically-cold
streams of stars that are extremely valuable tracers of the large- and
small-scale distribution of mass in the Galaxy.
Most stellar streams discovered around the Milky Way reside in the Galactic halo
and are therefore primarily sensitive to the dark matter distribution.
We present a sample of highly-probable members of the longest cold stream, GD-1,
selected to have small parallaxes and retrograde proper motions using data from
the \gaia\ mission, and \pans\ photometry consistent with an old and metal-poor
population at a distance of $\sim10~\kpc$.
This selection extended the stream by $10^\circ$(?), revealed a possible location of the progenitor and mapped significant density variations along the stream in unprecedented detail.
In addition to several prominent gaps, for the first time we also find evidence of stream members off the main track.
No(?) secular formation mechanisms predict the existence of such spurs, but they are a natural result of a stream interaction with a massive perturber.
Based on the impulse approximation arguments, we estimate that the detected underdensity and the associated spur could have been induced by a perturbation of a $\sim10^7\,\rm M_\odot$(?) object $\sim 0.5\,$Gyr ago.
Our model of GD-1 indicates that the stream crosses the Galactic plane at a large radius, which puts the likely origin of the perturber beyond the stellar disk.
\end{abstract}

\keywords{Galaxy: halo --- dark matter}

\section{Introduction}
\label{sec:intro}

Dynamically cold stellar streams are formed from the tidal disruption of stellar
systems by the gravitational field of their host galaxy.
The phase-space density and mean track of stars in streams therefore encode
information about the underlying distribution and shape of mass on galactic
scales (e.g., \citealt{Johnston:1999, Bonaca:2018}).
Most stellar streams are found in the halos of galaxies and are therefore
important tracers of the large-scale distribution of dark matter in galactic
halos.

Long, thin stellar streams are also excellent laboratories for studying
small-scale structures in the mass distribution:
thin streams can remain coherent for tens of orbital periods, depending in
detail on the symmetries of the underlying galactic mass distribution (e.g.,
\citealt{Erkal:2016a}).
Encounters between massive perturbers and stream stars imprint variations in the
phase-space density that can persist and remain visible for close to their total
survival time (e.g., \citealt{Yoon:2011}).
In this sense, streams represent one of the most promising directions for
studying or ruling out the existence of small-scale dark matter sub-halos, as
would be predicted by standard $\Lambda$ Cold Dark Matter ($\Lambda$CDM) theory
(\citealt{Erkal:2015, Sanders:2016, Bovy:2017}).

Over 30 candidate stellar streams have been discovered throughout the halo of
the Milky Way, thanks to large-area, multi-band photometric surveys (especially
the Sloan Digital Sky Survey, SDSS; \citealt{York:2000}).
The Milky Way streams have a wide range of properties such as length, velocity
dispersion, and density (e.g., \citealt{Odenkirchen:2001, Grillmair:2006,
Grillmair:2006b, Belokurov:2006, Belokurov:2007, Bonaca:2012, Shipp:2018}; see
also the recent review \citealt{Grillmair:2016, Newberg:2016}), and the majority
have no obvious, bound progenitor systems;
of the thin, likely globular-cluster-origin streams, only two have clear
progenitors (NGC 5466 and Palomar 5).

The most prominent and longest (in angular extent) thin stream is the GD-1
stream (named after its discoverers; \citealt{Grillmair:2006}).
GD-1 was found using a matched filter on photometric data from the SDSS and
was found to span $\approx 60^\circ$ in Heliocentric sky coordinates, partially
because of its relative proximity to the sun ($\approx 7$--$10~\textrm{kpc}$).
Despite being close and relatively high surface-density, no remnant progenitor
cluster or stellar system has been found associated with the stream, but its
width,  metallicity ($\feh \approx -1.4$; \citealt{Koposov:2010}) and estimated
stellar mass ($M \approx 2 \times 10^4~\msun$; \citealt{Koposov:2010}) are
consistent with being a disrupted globular cluster.

The long physical length of the GD-1 stream ($\sim 15~\kpc$) and location in the
Galactic halo (pericentric distance $r_\textrm{peri} \sim 13~\kpc$) make it an
ideal object for constraining the gravitational potential of dark matter in the
inner Milky Way.
Indeed, both from fitting orbits to binned phase-space measurements along the
GD-1 stream (\citealt{Koposov:2010}) and from modeling the stream track in
phase-space coordinates (\citealt{Bovy:2016}), it has been used to show that the
dark matter distribution within its orbit is consistent with being close to
spherical.

Its long length --- and therefore its implied old dynamical age --- combined
with its particular orbit also make it a prime stream to search for interactions
with dark matter subhalos:
because of its large pericentric distance and retrograde orbit with respect to
the Galactic bar, density variations in the stream are not expected from either
interactions with giant molecular clouds (\citealt{Amorisco:2016}) or resonant
encounters with the bar (\citealt{Pearson:2017}).
By estimating a stream age of $\approx 4~\textrm{Gyr}$ assuming a progenitor
mass of $10^5~\msun$, and by taking a subhalo mass function and density from
XXX, \citet{Erkal:2016} predict that GD-1 could have up to one significant, wide
($\approx 5$--$7^\circ$) gap caused by an interaction with a
$10^6$--$10^7~\msun$ subhalo.

\todo{hints of density variations} from initial work Koposov:2010, later confirmed by density modeling, but all low significance
(\citealt{Carlberg:2013}).
More recently, deep photometry of GD-1 from CFHT/Megacam has revealed
interesting small-scale ``wiggles'' and significant density variations along a $\approx 45^\citc$ span of the stream (\citealt{DeBoer:2017}):
[summarize their conclusions].

In this \article, we present clearest view of the GD-1 stream to date using
astrometric data from the \gaia\ mission data release 2 (\DR) combined with
precise photometry from the \pans\ survey.
We find XX under-densities gaps, some consistent with ...
We revisit...do not try to model or fit for Galactic potential.
Focus on stream properties, data.


\section{Data}
\label{sec:data}

We use astrometric data from the \gaia\ mission (\citealt{Prusti:2016}), data
release 2 (\citealt{Gaia-Collaboration:2018, Lindegren:2018}), and photometry
from the \pans\ survey, data release 1 (\citealt{Chambers:2016}) to select
high-confidence members of the GD-1 stream.

We retrieve \gaia\ data along the previously-identified track of the GD-1 stream
from the \gaia\ science archive\footnote{\url{https://gea.esac.esa.int/}} by
selecting all sources with small parallax, $\varpi < 1~\textrm{mas}$, and use a
set of spherical polygons to select batches of stars that have small latitude in
the heliocentric GD-1 coordinate system, $(\phi_1, \phi_2)$, defined in
\cite{Koposov:2010}.\footnote{This transformation is implemented using the
\package{Astropy} (\citealt{astropy}) coordinate transformation framework in the
\package{Gala} \texttt{Python} package (\citealt{gala}).}
We convert the sky coordinates and proper motions of the parallax-selected stars
to the GD-1 stream coordinate system, and correct the proper motions for solar
reflex motion by assuming that stars at a given stream longitude, $\phi_1$, have
a distance given by the linear relationship $d(\phi_1) = (0.05 \, \phi_1 +
10)~\textrm{kpc}$.
For the solar velocity in a Galactocentric rest frame, we use $\bs{v}_\odot =
(11.1, 232.24, 7.25)~\kms$ (\citealt{Schonrich:2010, Bovy:2015}).
\figurename~\ref{fig:selection} (top right) shows the distribution of stars with
$|\phi_2| < XX^\circ$ in solar-reflex-corrected proper motion components in the
GD-1 coordinate system, $\mu_{\phi_1, \odot}$ (which includes the $\cos{\phi_2}$
term) and $\mu_{\phi_2, \odot}$.
GD-1 member stars are visible as the over-density at approximately
$(\mu_{\phi_1, \odot}, \mu_{\phi_2, \odot}) \approx (XX, YY)~\masyr$.

As a first selection of GD-1 candidate member stars, we select stars with proper
motions $XX < \mu_{\phi_1, \odot} < XX~\masyr$ and $XX < \mu_{\phi_2, \odot} <
XX~\masyr$, as visualized by the rectangle in the top right panel of
\figurename~\ref{fig:selection}.
\figurename~\ref{fig:selection} (top left) shows all stars the pass the
selection on proper motion, plotted in the GD-1 coordinate system.
From kinematic selection alone, the stream is identifiable as the over-density
of stars around $\phi_2 = 0$ between longitudes $XX^\circ \lesssim \phi_1
\lesssim XX^\circ$.

\begin{figure}[h]
\begin{center}
\includegraphics[width=\textwidth]{stream-selection.pdf}
\end{center}
\caption{
    \todo{AB}
\label{fig:selection}
}
\end{figure}

To improve the contrast of the stream over the background, we next cross-match
the sample to the \pans\ photometric catalog and use the photometry to further
clean the sample.
\figurename~\ref{fig:selection} (bottom right) shows the distribution of
proper-motion-selected candidate stream stars in a de-reddened \pans\
color-magnitude diagram (CMD).
\todo{Did you also cut on $phi_2$?}.
The over-plotted isochrone is from the MESA Isochrones \& Stellar Tracks (MIST;
\citealt{Dotter:2016, Choi:2016, Paxton:2011}) and represents a
$12~\textrm{Gyr}$ old population with $\feh = -1.XX$ at a distance of $XX~\kpc$.
We use this isochrone to motivate a polygonal selection in de-reddened $g-i$
color and apparent $g$-band magnitude, as visualized by the \todo{shaded} region
in the bottom right panel of \figurename~\ref{fig:selection}.

\figurename~\ref{fig:selection} (bottom left) shows the final sample of GD-1
stream member candidates after selecting on both proper motions and photometry.
\todo{Comment on how this is the bomb ass diggity and the clearest view of the
stream so far}
\todo{Mention the under-densities}
\todo{Already mention the spur here}

All of the clearly identifiable portions of the GD-1 stream are located at high
Galactic latitudes ($b > 20^\circ$), and we therefore do not expect significant
dust extinction or variations in extinction along the stream.
\figurename~\ref{fig:sfd-cmd} (top panel) shows the $V$-band extinction in the
region around the GD-1 stream, computed from the Schlegel-Finkbeiner-Davis
extinction map (\cite{Schlegel:1998}; hereafter SFD).
For stream longitudes $-60^\circ < \phi_1 < 10^\circ$ and $-1^\circ < \phi_2 <
1^\circ$, the median $V$-band extinction is $\textrm{med}\left(A_V\right) =
0.04~\mag$, and the dispersion in $A_V$ in this region is $\sigma_{A_V} \approx
0.03~\textrm{mag}$.
For longitudes $\phi_1 < -60^\circ$, as the stream approaches the Galactic disk,
dust extinction becomes more appreciable.

Another \todo{segue to this paragraph...}.
However, we find that \todo{...}.
We discuss the scanning pattern in more detail in
Appendix~\ref{sec:completeness}.


\section{Results}
\label{sec:results}

\todo{segue sentences...}

\subsection{Global properties}
\label{sec:res_global}

From the combined proper motion and CMD selection of stream members
(\sectionname~\ref{sec:data}), it is now clear that the GD-1 stream extends
at least $90^\circ$ in apparent length.
Several clear under-densities and gaps are also
\figurename~\ref{fig:sfd-cmd} (middle panel) again shows the proper-motion and CMD selected stream stars, with several features

\begin{figure}[h]
\begin{center}
\includegraphics[width=\textwidth]{sfd-cmd.pdf}
\end{center}
\caption{%
\todo{AB}
\label{fig:sfd-cmd}
}
\end{figure}


\subsection{Gaps and under-densities}
\label{sec:res_gap}

\begin{figure}[h]
\begin{center}
\includegraphics[width=\textwidth]{track-and-model.pdf}
\end{center}
\caption{%
\todo{AB}
\label{fig:track-and-model}
}
\end{figure}


\subsection{Spur}
\label{sec:res_spur}

\begin{figure}[h]
\begin{center}
\includegraphics[width=\textwidth]{spur.pdf}
\end{center}
\caption{%
\todo{AB}
\label{fig:spur}
}
\end{figure}


\section{Discussion}
\label{sec:discussion}


\section{Conclusions}
\label{sec:conclusions}


\acknowledgements{
It is a pleasure to thank
Vasily Belokurov,
Andrew R. Casey,
Marla Geha,
David W. Hogg,
Kathryn V. Johnston,
Sergey Koposov,
Mariangela Lisanti,
Edward Schlafly,
and David N. Spergel for useful discussions and feedback.

This work has made use of data from the European Space Agency (ESA) mission {\it
Gaia} (\url{https://www.cosmos.esa.int/gaia}), processed by the {\it Gaia} Data
Processing and Analysis Consortium (DPAC,
\url{https://www.cosmos.esa.int/web/gaia/dpac/consortium}). Funding for the DPAC
has been provided by national institutions, in particular the institutions
participating in the {\it Gaia} Multilateral Agreement.  This research was
started at the NYC Gaia DR2 Workshop at the Center for Computational
Astrophysics of the Flatiron Institute in 2018 April.

AB acknowledges generous support from the Institute for Theory and Computation
at Harvard University.
All code used in this work and all results are available at
\url{https://github.com/adrn/GD1-DR2}.
}

\software{
    \package{Astropy} \citep{astropy},
    \package{dustmaps}\footnote{\url{https://github.com/gregreen/dustmaps}},
    \package{gala} \citep{gala},
    \package{IPython} \citep{ipython},
    \package{matplotlib} \citep{mpl},
    \package{numpy} \citep{numpy},
    \package{scipy} \citep{scipy}
}

\bibliographystyle{aasjournal}
\bibliography{gd1}

\clearpage

\appendix
\section{Completeness and the \gaia\ scanning pattern}
\label{sec:completeness}

% % Notebook:
% \begin{figure}[h]
% \begin{center}
% \includegraphics[width=0.7\textwidth]{nvisits.pdf}
% \end{center}
% \caption{%
% TODO
% \label{fig:TODO}
% }
% \end{figure}


\end{document}
